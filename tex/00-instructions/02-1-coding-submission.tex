{\bf Coding Submission:}
All assignment code is in the |src/| subirectory.  You will submit only the
|src/submission.py| file.  Please only make changes between the lines containing
|### START_CODE_HERE ###| and |### END_CODE_HERE ###|. Do not make changes to
% The astute reader will notice that there are underscores in the
% START_CODE_HERE and END_CODE_HERE flags, while the source files actually
% contain spaces.  This is because our solution sanitization script will
% otherwise recognize this tag and attempt to delete it.  The underscores are a
% simple way to prevent this.
files other than |src/submission.py|.

The unit tests in |src/grader.py| will be used to autograde your submission.
Run the autograder locally using the following terminal command within the
|src/| subdirectory:
\begin{lstlisting}
python grader.py
\end{lstlisting}

There are two types of unit tests used by our autograders:
\begin{itemize}
  \item |basic|:  These unit tests will verify only that your code runs without
  errors on obvious test cases.
  \item |hidden|: These unit tests will verify that your code produces correct
  results on complex inputs and tricky corner cases.  In the student version of
  |src/grader.py|, only the setup and inputs to these unit tests are provided.
  When you run the autograder locally, these test cases will run, but the
  results will not be verified by the autograder.
\end{itemize}

For debugging purposes, a single unit test can be run locally.  For example, you
can run the test case |3a-0-basic| using the following terminal command within
the |src/| subdirectory:
\begin{lstlisting}
python grader.py 3a-0-basic
\end{lstlisting}

Before beginning this course, we highly recommend you walk through our
\href{https://github.com/scpd-proed/General_Handouts/blob/master/Anaconda_Setup.pdf}{Anaconda
Setup for XCS Courses} to familiarize yourself with
our coding environment.  Please use the env defined in |src/environment.yml|
to run your code.  This is the same environment used by our autograder.
